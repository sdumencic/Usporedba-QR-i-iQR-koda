\chapter{Osnovna svojstva QR kodova}





\section{Struktura QR koda}
Kao što je spomenuto već u uvodu, svaki QR kod sastoji se od matrice u kojoj se nalaze crni i bijeli moduli - kvadratići. Prema smještaju tih modula, čitač QR koda prepoznaje o kojim podatcima se radi. Ta matrica je okružena bijelim obrubom koji se naziva tiha zona (eng. Quiet Zone). QR kod se sastoji od sljedećih dijelova:

\begin{itemize}
 \setlength\itemsep{1ex} % prilagodjava vertikalni razmak izmedju stavki u listi. Prilagodite 0.5 vasem ukusu.
    \item Područje kodiranja (eng. Encoding Region)
    \item Tiha zona (eng. Quiet Zone)
    \item Funkcijski uzorci (eng. Function Patterns)
\end{itemize}

\vspace{5mm} %5mm vertical space

\setlength{\parindent}{0ex}Funkcijski uzorci (eng. Function Patterns) QR kodova moraju biti smješteni u točno određenim dijelovima QR koda kako bi osigurali mogućnost prepoznavanja orijentacije kod čitanja QR koda. Postoje 4 funkcijska uzorka. Funkcijski uzorci su:
\begin{itemize}
	    \begin{itemize} \item Uzorci pretraživača (eng. Finder Patterns) 
	    \item separatori (eng. Separators)
	    \item Vremenski uzorci (eng.Timing Patterns)
	    \item Uzorci poravnanja (eng. Alignment Patterns)
	    \end{itemize}
    \end{itemize}
\newpage
\subsection{Uzorci pretraživača (eng. Finder Patterns)}
\setlength{\parindent}{1cm}Ovaj uzorak je najuočljiviji kad se gleda QR kod. Prikazan je trima kvadratima koji se nalaze u trima uglovima matrice (gornji lijevi, gornji desni i donji lijevi ugao). \\
Ovaj uzorak služi za orijentaciju čitanja koda.

\begin{figure}[!htpb]
	\begin{center}
	\includegraphics[width=5cm,keepaspectratio=true]{images/qr_finder_pattern.PNG}
	   
    \caption{Lokacija uzoraka pretraživača}
    \label{fig:ID_slike}
	\end{center}
\end{figure}

\subsection{Separatori (eng. Separators)}
\setlength{\parindent}{1cm}Separator je bijeli obrub oko uzorka pretraživača koji odjeljuje uzorke pretraživača od ostatka matrice.

\begin{figure}[!htpb]
	\begin{center}
	\includegraphics[width=5cm,keepaspectratio=true]{images/separator.PNG}
	   
    \caption{Lokacija separatora}
    \label{fig:ID_slike}
	\end{center}
\end{figure}

\subsection{Vremenski uzorci (eng. Timing Patterns)}
\setlength{\parindent}{1cm}U QR kodu nalaze se 2 vremenska uzorka: vertikalni i horizontalni. Sastoje se od bijelih i crnih modula i povezuju separatore. Vertikalni vremenski uzorak spaja separatore koji se nalaze oko gornjeg lijevog uzorka pretraživača i donjeg lijevog uzorka pretraživača. Horizontalni vremenski uzorak povezuje separatore koji se nalaze oko gornjeg lijevog i gornjeg desnog uzorka pretraživača. \\
Ovi uzorci služe za određivanje  koordinata modula i informacije o verziji QR koda.

\subsection{Uzorci poravnanja (eng. Alignment Patterns)}
Uzorak poravnanja nalazi se u donjem desnom dijelu matrice. Sve verzije QR kodova nakon QR koda tipa 2 moraju sadržavati ovaj uzorak te njihov broj ovisi o verziji.

\begin{figure}[!htpb]
	\begin{center}
	\includegraphics[width=5cm,keepaspectratio=true]{images/alignment.PNG}
	   
    \caption{Lokacija uzoraka poravnanja}
    \label{fig:ID_slike}
	\end{center}
\end{figure}

\subsection{Područje kodiranja (eng. Encoding region)}
\setlength{\parindent}{1cm}Područje kodiranja služi za pohranu informacija o formatu i verziji, podatke i kodove za ispravljanje grešaka.
Za informacije o formatu rezervirano je područje oko uzoraka pretraživača.
Informacija o verziji nalazi se iznad donjeg lijevog uzorka pretraživača i lijevo od gornjeg desnog uzorka pretraživača.

\subsection{Tiha zona (eng. Quiet Zone)}
Zona koja ne sadrži podatke. Služi za odvajanje QR koda od ostalih uzoraka na površini na kojoj se nalazi kako ne bi došlo do pogreške prilikom čitanja koda.

\newpage

\section{Ispravljanje grešaka i oštećenja}
\setlength{\parindent}{1cm}QR kodovi omogućuju ispravljanje grešaka (eng. error correction). To je omogućeno time što se u ključne riječi (codeword) podataka dodaju ključne riječi za ispravljanje pogrešaka. Na taj način prilikom čitanja koda, mogu se pročitati i kodovi koji su na neki način oštećeni. Matematička metoda kojom se postiže ispravljanje grešaka je pomoću  Reed-Solomon kodova. Također, postoje 4 različite razine ispravka oštećenja: 

Razina ispravka oštećenja bira se na temelju uvjeta na kojima se nalazi te veličine QR koda. Ovisno o tome gdje i za što se QR kod koristi postoji potrebna razina error correction tehnologije koju se treba ukomponirati u kod. 
%Pa tako primjerice za tvornice u kojima je velika vjerojatnost da će se kod na neki način oštetiti ili zaprljati, potrebna je viša razina za ispravljanje grešaka, primjerice Q ili H razina. Najčešće se koristi razina M.

\begin{figure}[!htpb]
	\begin{center}
	\subfloat[Ispravljanje grešaka može se koristiti da bi se dizajnirao QR kod, jer zbog ispravljanja grešaka, slika koja se nalazi preko QR koda, ne ometa čitanje]{\label{fig:a} \includegraphics[width=5cm,keepaspectratio=true]{images/2_150_150DPI_ty_oerny_08_2011.jpg}}
	\hspace{1em}
	\subfloat[QR kod na slici je vidno oštećen tj. nedostaje dio koda te zbog ispravljanja pogrešaka to ne ometa čitanje koda]{\label{fig:b}  %\\ % ukoliko se hoce iducu sliku u novi red
    \includegraphics[width=5cm,keepaspectratio=true]{images/800px-QR_Code_Damaged.jpg}}\\
    \caption{Primjene ispravljanja grešaka}
    \label{fig:ID_slike}
	\end{center}
\end{figure}


\newpage
\section{Proces kodiranja i dekodiranja QR koda}

\subsection{Kodiranje QR koda}
Kodiranje QR koda obavlja se u nekoliko koraka:

\begin{enumerate}
 \setlength\itemsep{1ex} 
    \item Analiza podataka (eng. Data Analysis)
    \item Kodiranje podataka (eng. Data encoding)
    \item Kodiranje koda za ispravljanje grešaka (eng. Error Correction Coding)
    \item Struktura završne poruke (eng. Structure Final Message)
    \item Pozicioniranje modula u matricu (eng. Module Placement in Matrix)
    \item Maskiranje podataka (eng. Data Masking)
    \item Informacije o formatu i verziji (eng. Format and Version Information)
\end{enumerate}

\subsubsection{Analiza podataka (eng. Data Analysis)}
QR standard ima 4 standardizirana načina za kodiranje teksta, a to su numerički, alfanumerički, bajtovi i Kanji \footnote{kineski znakovi korišteni u modernom japanskom jeziku}. Zajedničko svim standardiziranim načinima kodiranja je da kodiraju tekst kao niz bitova. Svaki način koristi drugačiju metodu prilikom pretvaranja teksta u bitove.

Svaki od načina je optimiziran za kodiranje podataka u najkraćem mogućem nizu bitova. Prema tome, ovaj korak služi za prepoznavanje može li se tekst kodirati na neki od ovih načina i ako može, prepoznavanje i odabir načina koji je optimalan za tekst.

Indikator načina (eng. Mode indicator) bira koji će se način koristiti.

\subsubsection{Kodiranje podataka (eng. Data encoding)}
Ovaj korak odnosi se na kodiranje teksta koji se prikazuje kao niz bitova. Podatci su odijeljeni u nizove od 8 bitova, dakle 1 bajt. 

Indikator načina kodiranja, spomenut u prošlom koraku, je niz od 4 bita te svaki kodirani podatak mora započeti s odgovarajućim indikatorom načina.

Također, postoji indikator za brojanje znakova (eng. Character Count Indicator) koji služi za brojanje znakova koje se kodira. On se postavlja nakon indikatora načina kodiranja i njegova duljina ovisi o verziji QR koda. 

\subsubsection{Kodiranje koda za ispravljanje grešaka (eng. Error Correction Coding)}
Za ispravljanje grešaka koristi se već spomenuta matematička metoda ispravljanja grešaka Reed-Solomon. Ta metoda koristi ključne riječi dobivene iz nizova bitova teksta kojeg želimo kodirati.

Čitači QR kodova čitaju i ključne riječi podataka i ključne riječi za ispravljanje grešaka. Zatim čitač  uspoređuje nizove bitova ova dva dijela i zaključuje je li došlo do greške. Ukoliko je došlo do greške, čitač ju ispravlja i može pročitati QR kod.

\subsubsection{Struktura završne poruke (eng. Structure Final Message)}
Kodirani podatci se organiziraju u ispravan redoslijed.

\subsubsection{Pozicioniranje modula u matricu (eng. Module Placement in Matrix)}
Nakon postavljanja ključnih riječi u ispravan redoslijed, bitovi se postavljaju u QR kod matricu na određen način. 

\subsubsection{Maskiranje podataka (eng. Data Masking)}
Ovaj korak je potreban jer čitači QR koda mogu imati problema prilikom čitanja nekih uzoraka. Da bi se riješio taj problem, QR kod sadrži masku koja mijenja QR kod ovisno o uzorku.

\subsubsection{Informacije o formatu i verziji (eng. Format and Version Information)}
U QR kod se dodaje informacije o formatu i ukoliko je potrebno o verziji. To se radi na način da se u QR kod dodaju pikseli u određena područja matrice koja su nakon prethodnih koraka ostala nepopunjena. Pikseli vezani za format označavaju razinu ispravljanja grešaka i maske koje QR kod koristi. Pikseli vezani uz verziju se koriste samo u većim QR kodovima i  kodiraju veličinu matrice.

\newpage

\subsection{Dekodiranje QR koda}
Dekodiranje QR koda obavlja se u nekoliko koraka:

\begin{enumerate}
 \setlength\itemsep{1ex} 
    \item Prepoznavanje modula (eng. Recognizing Modules)
    \item Dohvaćanje informacija o formatu (eng. Extract Format Information)
    \item Određivanje informacija o verziji (eng. Determine Version Information)
    \item Aktivacija maske (eng. Release Masking)
    \item Uspostavljanje kodnih riječi za ispravljanje pogrešaka (eng. Restore Data and Error Correction Codewords)
    \item Detekcija i ispravljanje grešaka (eng. Error Detection and Correction)
    \item Dekodiranje kodnih riječi (eng. Decode Data Codewords)
\end{enumerate}

\subsubsection{Prepoznavanje modula (eng. Recognizing Modules)}

Čitač QR koda razlikuje crne i bijele module kao binarne znamenke "1" ili "0".

\subsubsection{Dohvaćanje informacija o formatu (eng. Extract Format Information)}

Primjenjuje informacije o formatu na module od kojih je sačinjena matrica QR koda.

\subsubsection{Određivanje informacija o verziji (eng. Determine Version Information)}

Prema djelu QR koda gdje se nalaze informacije o verziji određuje o kojoj verziji QR koda se radi.

\subsubsection{Aktivacija maske (eng. Release Masking)}
U ovom dijelu procesa dekodiranja čitač QR koda primjenjuje masku dobivenu iz informacija o formatu, na područje kodiranja (eng. Encoding region).





\subsubsection{Uspostavljanje kodnih riječi za ispravljanje pogrešaka (eng. Restore Data and Error Correction Codewords)}

Iz matrice uspostavlja kodne riječi za ispravak grešaka.

\subsubsection{Detekcija i ispravljanje grešaka (eng. Error Detection and Correction)}

Ukoliko čitač QR koda otkrije grešku ili oštećenje koda, ispravlja ih kako bi mogao ispravno pristupiti podatcima.

\subsubsection{Dekodiranje kodnih riječi (eng. Decode Data Codewords)}

Kodne riječi se dijele u skupine prema kojima čitač QR koda zatim ispisuje podatak koji je bio pohranjen u QR kodu.



