\chapter{Osnovna svojstva QR kodova}

\section{Pohrana podataka}
Već u uvodu je spomenuto da postoje različiti oblici QR kodova, međutim standardni oblik QR koda također ima različite verzije. Verzije QR koda sežu od verzije 1 do verzije 40. Svaka od verzija ima svoj izgled koji se razlikuje po količini modula od kojih je sastavljena matrica QR koda. Matrica QR koda verzije 1 sastoji se od matrice veličine 21 x 21 modul, dok se verzija 40 sastoji od matrice veličine 177 x 177 modula što upućuje na veću količinu podataka koja se može pohraniti u verziji 40. 

\section{Ispravljanje grešaka i oštećenja}
QR kodovi omogućuju ispravljanje grešaka (eng. error correction). To je omogućeno time što se u ključne riječi (codeword) podataka dodaju ključne riječi za ispravljanje pogrešaka. Na taj način prilikom čitanja koda, mogu se pročitati i kodovi koji su oštećeni na neki način. Matematička metoda kojom se postiže ispravljanje grešaka je pomoću  Reed-Solomon kodova. Također, postoje 4 različite razine ispravka oštećenja: 

\newpage
%%%%%%%%% tabela
\renewcommand{\arraystretch}{1.2}  % prilagodjava vertikalni razmak izmedju redaka cijele tabele
\begin{table}[!htbp]
\caption{Razine ispravka oštećenja}
\centering
\begin{tabular}{|c|c|c|}
\hline
Redni broj razine za ispravak & Razina & Približan postotak koda koji se može ispraviti \\ [0.5ex]  % oblik [0.5ex] je pojedinacni oblik reguliranja vertikalnoga razmaka izmedju redataka tabele koji vrijedi za specificno mjesto na kojem je navedeno. (Bolje je koristiti \arraystretch oblik prije tabele.)
\hline \hline
1. & L & 7\% \\ \hline
2. & M & 15\% \\ \hline
3. & Q & 25\% \\ \hline
4. & H & 30\% \\\hline
\end{tabular}
\label{tab:ID_tabele}
\end{table}

Razina ispravka oštećenja bira se na temelju uvjeta na kojima se nalazi te veličine QR koda. Ovisno o tome gdje i za što se QR kod koristi postoji potrebna razina error correction tehnologije koju se treba ukomponirati u kod. Pa tako primjerice za tvornice u kojima je velika vjerojatnost da će se kod na neki način oštetiti ili zaprljati, potrebna je viša razina za ispravljanje grešaka, primjerice Q ili H razina. Najčešće se koristi razina M.

\subsection{Izgled QR koda}
Minimalne dimenzije QR koda su $11 \times $11 modula što u slučajevima kad se sprema dovoljno malen broj podataka rezultira u neiskorištenom prostoru.
Maksimalne dimenzije QR koda su $171 \times $171, to omogućava pohranu do 7 089 znakova.


\section{Struktura QR koda}
Kao što je spomenuto već u uvodu, svaki QR kod sastoji se od matrice u kojoj se nalaze crni i bijeli moduli - kvadratići. Prema smještaju tih modula, čitač QR koda prepoznaje o kojim podatcima se radi. Ta matrica je okružena bijelim obrubom koji se naziva tiha zona (eng. Quiet Zone). Funkcijski uzorci (eng. Function Patterns) QR kodova moraju biti smješteni u točno određenim dijelovima QR koda kako bi osigurali mogućnost prepoznavanja orijentacije kod čitanja QR koda. Postoje 4 funkcijska uzorka. QR kod se sastoji od slijedećih dijelova:

\begin{itemize}
 \setlength\itemsep{1ex} % prilagodjava vertikalni razmak izmedju stavki u listi. Prilagodite 0.5 vasem ukusu.
	\item Funkcijski uzorci:\begin{itemize}
	    \item Uzorci pretraživača (eng. Finder Patterns) 
	    \item separatori (eng. Separators)
	    \item Vremenski uzorci (eng.Timing Patterns)
	    \item Uzorci poravnanja (eng. Alignment Patterns)
\end{itemize}
    \item Područje kodiranja (eng. Encoding Region)
    \item Tiha zona (eng. Quiet Zone)
\end{itemize}



\subsection{Uzorci pretraživača (eng. Finder Patterns)}
Ovaj uzorak je najuočljiviji kad se gleda QR kod. Prikazan je trima kvadratima koji se nalaze u trima uglovima matrice (gornji lijevi, gornji desni i donji lijevi ugao). \\
Ovaj uzorak služi za orijentaciju čitanja koda.

\begin{figure}[!htpb]
	  \begin{center}
	    \includegraphics[width=5cm,keepaspectratio=true]{images/qr_finder_pattern.PNG}
	   
\caption{Lokacija uzoraka pretraživača}
\label{fig:ID_slike}
	  \end{center}
\end{figure}

\subsection{Separatori (eng. Separators)}
Separator je bijeli obrub oko uzorka pretraživača koji odjeljuje uzorke pretraživača od ostatka matrice.

\begin{figure}[!htpb]
	  \begin{center}
	    \includegraphics[width=5cm,keepaspectratio=true]{images/separator.PNG}
	   
\caption{Lokacija separatora}
\label{fig:ID_slike}
	  \end{center}
\end{figure}
\newpage
\subsection{Vremenski uzorci (eng. Timing Patterns)}
U QR kodu nalaze se 2 vremenska uzorka: vertikalni i horizontalni. Sastoje se od bijelih i crnih modula i povezuju separatore. Vertikalni vremenski uzorak spaja separatore koji se nalaze oko gornjeg lijevog uzorka pretraživača i donjeg lijevog uzorka pretraživača. Horizontalni vremenski uzorak povezuje separatore koji se nalaze oko gornjeg lijevog i gornjeg desnog uzorka pretraživača. \\
Ovi uzorci služe za određivanje  koordinata modula i informacije o verziji QR koda.

\subsection{Uzorci poravnanja (eng. Alignment Patterns)}
Uzorak poravnanja nalazi se u donjem desnom dijelu matrice. Sve verzije QR kodova nakon QR koda tipa 2 moraju sadržavati ovaj uzorak te njihov broj ovisi o verziji.

\begin{figure}[!htpb]
	  \begin{center}
	    \includegraphics[width=5cm,keepaspectratio=true]{images/alignment.PNG}
	   
\caption{Lokacija uzoraka poravnanja}
\label{fig:ID_slike}
	  \end{center}
\end{figure}

\subsection{Područje kodiranja (eng. Encoding region)}
Područje kodiranja služi za pohranu informacija o formatu i verziji, podatke i kodove za ispravljanje grešaka.
Za informacije o formatu rezervirano je područje oko uzoraka pretraživača.
Informacija o verziji nalazi se iznad donjeg lijevog uzorka pretraživača i lijevo od gornjeg desnog uzorka pretraživača.

\subsection{Tiha zona (eng. Quiet Zone)}
Zona koja ne sadrži podatke. Služi za odvajanje QR koda od ostalih uzoraka na površini na kojoj se nalazi kako nebi došlo do pogreške prilikom čitanja koda.

\newpage

\section{Proces kodairanja i dekodiranja QR koda}

\subsection{Kodiranje QR koda}
Kodiranje QR koda obavlja se u nekoliko koraka:

\begin{enumerate}
 \setlength\itemsep{1ex} 
    \item Analiza podataka (eng. Data Analysis)
    \item Kodiranje podataka (eng. Data encoding)
    \item Kodiranje koda za ispravljanje grešaka (eng. Error Correction Coding)
    \item Struktura završne poruke (eng. Structure Final Message)
    \item Pozicioniranje modula u matricu (eng. Module Placement in Matrix)
    \item Maskiranje podataka (eng. Data Masking)
    \item Informacije o formatu i verziji (eng. Format and Version Information)
\end{enumerate}

\subsubsection{Analiza podataka (eng. Data Analysis)}
QR standard ima 4 standardizirana načina za kodiranje teksta, a to su numerički, alfanumerički, bajtovi i Kanji \footnote{kineski znakovi korišteni u modernom japanskom jeziku}. Zajedničko svim standardiziranim načinima kodiranja je da kodiraju tekst kao niz bitova. Svaki način koristi drugačiju metodu prilikom pretvaranja teksta u bitove.

Svaki od načina je optimiziran da kodira podatke u najkraćem mogućem nizu bitova. Prema tome, ovaj korak služi za prepoznavanje može li se tekst kodirati na neki od ovih načina te odabiru načina koji je najoptimalniji za tekst.

Indikator načina (eng. Mode indicator) bira koji će se način koristiti.

\subsubsection{Kodiranje podataka (eng. Data encoding)}
Ovaj korak odnosi se na kodiranje teksta koji se prikazuje kao niz bitova. Podatci su odijeljeni u nizove od 8 bitova, dakle 1 bajt. 

Indikator načina kodiranja spomenut u prošlom koraku je niz od 4 bita te svaki kodirani podatci moraju započeti s odgovarajućim indikatorom načina.

Također postoji indikator za brojanje znakova (eng. Character Count Indicator) koji služi za brojanje znakova koje se kodira. On se postavlja nakon indikatora načina kodiranja i njegova duljina ovisi o verziji QR koda. 

\subsubsection{Kodiranje koda za ispravljanje grešaka (eng. Error Correction Coding)}
Za ispravljanje grešaka koristi se već spomenuta matematička metoda ispravljanja grešaka Reed-Solomon. Ta metoda koristi ključne riječi dobivene iz nizova bitova teksta kojeg želimo kodirati.

Čitači QR kodova čitaju i ključne riječi podataka i ključne riječi za ispravljanje grešaka. Zatim čitač  uspoređuje nizove bitova ta dva dijela i zaključuje je li došlo do greške. Ukoliko je došlo do greške, čitač ju ispravlja i može pročitati QR kod.

\subsubsection{Struktura završne poruke (eng. Structure Final Message)}
Kodirani podatci se organiziraju u ispravan redoslijed.

\subsubsection{Pozicioniranje modula u matricu (eng. Module Placement in Matrix)}
Nakon postavljanja ključnih riječi u ispravan redolsijed, bitovi se postavljaju u QR kod matricu na određen način. 

\subsubsection{Maskiranje podataka (eng. Data Masking)}
Ovaj korak je potreban jer čitači QR koda mogu imati problema prilikom čitanja nekih uzoraka. Da bi se riješio taj problem, QR kod sadrži masku koja mijenja QR kod ovisno o uzorku.

\subsubsection{Informacije o formatu i verziji (eng. Format and Version Information)}
U QR kod se dodaje informacije o formatu i ukoliko je potrebno o verziji. To se radi na način da se u QR kod dodaju pikseli u određena područja matrice koja su nakon prethodnih koraka ostala nepopunjena. Pikseli vezani za format označavaju razinu ispravljanja grešaka i maske koje QR kod koristi. Pikseli vezani uz verziju se koriste samo u većim QR kodovima i  kodiraju veličinu matrice.

\newpage

\subsection{Dekodiranje QR koda}
Dekodiranje QR koda obavlja se u nekoliko koraka:

\begin{enumerate}
 \setlength\itemsep{1ex} 
    \item Prepoznavanje modula (eng. Recognizing Modules)
    \item Dohvaćanje informacija o formatu (eng. Extract Format Information)
    \item Određivanje informacija o verziji (eng. Determine Version Information)
    \item nesto (eng. Release Masking)
    \item Uspostavljanje kodnih riječi za ispravljanje pogrešaka (eng. Restore Data and Error Correction Codewords)
    \item Detekcija i ispravljanje grešaka (eng. Error Detection and Correction)
    \item Dekodiranje kodnih riječi (eng. Decode Data Codewords)
\end{enumerate}

\subsubsection{Prepoznavanje modula (eng. Recognizing Modules)}

Čitač QR koda razlikuje crne i bijele module kao binarne znamenke "1" ili "0".

\subsubsection{Dohvaćanje informacija o formatu (eng. Extract Format Information)}

Primjenjuje informacije o formatu na module od kojih je sačinjenq matrica QR koda.

\subsubsection{Određivanje informacija o verziji (eng. Determine Version Information)}

Prema djelu QR koda gdje se nalaze informacije o verziji određuje o kojoj verziji QR koda se radi.

\subsubsection{nesto (eng. Release Masking)}
U ovom dijelu procesa dekodiranja čitač QR koda primjenjuje masku dobivenu iz informacija o formatu, na područje kodiranja (eng. Encoding region).





\subsubsection{Uspostavljanje kodnih riječi za ispravljanje pogrešaka (eng. Restore Data and Error Correction Codewords)}

Uzima/Uspostavlja kodne riječi za ispravak grešaka, iz matrice.

\subsubsection{Detekcija i ispravljanje grešaka (eng. Error Detection and Correction)}

Ukoliko čitač QR koda otkrije grešku ili oštećenje koda, ispravlja ih kako bi mogao ispravno pristupiti podatcima.

\subsubsection{Dekodiranje kodnih riječi (eng. Decode Data Codewords)}

Kodne riječi se dijele u skupine prema kojima čitač QR koda zatim ispisuje podatak koji je bio pohranjen u QR kodu.

\section{Primjena}
Osim prvotne ideje za primjenu QR koda - praćenje automobilsih dijelova u tvornici, QR kod se proširio na zaista veliko područje korištenja/primjene. U svakodnevnom životu, QR kodovi su se počeli koristiti početkom desetljeća kada su ga najviše koristili Amerikanci. Primjena u to doba bila je znazno teža nego danas s obzirom na manji broj ljudi koji su posjedovali pametne telefone. S razvojem tehnologije razvijao se i ovaj oblik tehnologije te je postao opće prihvaćen. 

Jedan od primjera primjene je modernizacija putovanja prijevoznim sredstvima je korištenje QR koda koji možemo imati na mobitelu te se koristi umjesto fizičke putne karte. Ovakvih primjena je sve više jer su lako dostupne s obzirom na razvoj i dostupnost tehnologije koja je postala svakodnevni dio naših života. 

Također jedan on načina korištenja QR kodova je pohrana informacija poput linka na primjerice, proizvodima, mjestima u gradu itd. S obzirom da većina ljudi razvijenih zemalja posjeduje mobitel, skeniranjem tih kodova se lako dođe do potrebnih informacija o mjestu na kojem se nalazimo, proizvodu koji kupujemo...

Jedna primjena koja postaje sve popularnija je i plaćanje računa skeniranjem QR koda.

Prema istraživanjima 

\chapter{Standardni QR kod}

\chapter{iQR kodovi}
\section{Općenito o iQR kodovima}

iQR kodovi su, također kao i QR kodovi, 2D matrice koje omogućuju pohranu podataka te pomoći čitača, dekodiranje te matrice i pristup onome što je pohranjeno u kodu. 

iQR kodovi nastali su kao potreba za pohranjivanjem većeg broja podataka na manjoj površini. Na prvi pogled razlika možda i nije toliko očita, ali postoji. Glavna razlika je u tome što kada se uzme QR kod i iQR kod jednakih dimenzija, u iQR kod se može pohraniti 80\% više podataka nego u QR kod. Također, za jednaku količinu podataka koju želimo spremiti u QR i iQR kod, iQR kod bi bio manjih dimenzija nego QR kod, točnije, iQR kod bio bi 30\% manji nego QR kod.

\section{Izgled}
Minimalne dimenzije iQR koda je $9 \times $9 modula. To omogućava optimalnije korištenje prostora s obzirom da se za manju količinu podataka može koristiti manje dimenzije nego kod QR kodova.
Maksimalna veličina iQR koda je $422 \times $422 modula, to omogućava pohranu do \newline 40 637 znakova.

Prednost iQR koda su i u tome što ne mora nužno biti kvadratnog oblika, nego može biti i pravokutnik. S obzirom da je primjena kodova široka, ovo omogućava da se iQR kod optimalnije namjesti na poziciju na predmetu na kojem se nalazi. Primjerice, na zaobljene predmete u obliku primjerice valjda, lakše je uzduž postavit pravokutni iQR kod, nego kvadratni, jer će biti lakše za očitat s obzirom da zbog zakrivljenosti predmeta neće doći do toliko velike deformacije koja bi uzrokovala otežano čitanje koda.

Međutim, korištenje pravokutnog koda, smanjuje kapacitet podataka koje se može pohraniti u matricu iQR koda. Tako maksimalne dimenzije postaju $43 \times $131 modula, a maksimalna pohrana je 1 202 znaka.

\section{Ispravljanje grešaka i oštećenja}

iQR kodovi omogućavaju ispravljanje grešaka u 6 razina. Međutim treba pametno odabrati razinu ispravka s obzirom da veća razina znači da je manja maksimalna količina podataka koju možemo pohraniti u kod. 
\renewcommand{\arraystretch}{1.2}  % prilagodjava vertikalni razmak izmedju redaka cijele tabele
\begin{table}[!htbp]
\caption{Razine ispravka oštećenja}
\centering
\begin{tabular}{|c|c|c|}
\hline
Redni broj razine za ispravak & Razina & Približan postotak koda koji se može ispraviti \\ [0.5ex]  % oblik [0.5ex] je pojedinacni oblik reguliranja vertikalnoga razmaka izmedju redataka tabele koji vrijedi za specificno mjesto na kojem je navedeno. (Bolje je koristiti \arraystretch oblik prije tabele.)
\hline \hline
1. & L & 7\% \\ \hline
2. & M & 15\% \\ \hline
3. & Q & 25\% \\ \hline
4. & H & 30\% \\\hline
5. & S & 50\% \\ \hline
6. & T & 60\% \\ \hline
\end{tabular}
\label{tab:ID_tabele}
\end{table}

%Minimalne dimenzije QR koda su 11 \times 11 modula što u slučajevima kad se sprema dovoljno malen broj podataka rezultira u neiskorištenom prostoru.