\chapter*{Zaključak} 
\addcontentsline{toc}{chapter}{Zaključak}

QR kodovi su sami po sebi velik napredak u odnosu na barkodove (eng. Barcode)\footnote{Jednodimenzionalni kodovi za pohranu podataka.} koji su se ranije koristili. Razlog tome je to što se QR kodovi mogu očitavati i s papira i s ekrana, dok se barkodovi mogu očitati samo s papira. Velika je prednost i količina informacija koja se može pohraniti u QR kodove.

Razvoj QR kodova doveo je do nekoliko verzija QR kodova koji se razlikuju u veličini matrice, a s time i u kapacitetu pohrane podataka. Također je iz ideje standardnog QR koda osmišljen i iQR kod, koji je u svojim značajkama napredniji.

Usporedbom izgleda, jasno je da QR kod treba više prostora za jednaku količinu podataka koju sadržava iQR kod. Također, ukoliko su matrice QR i iQR koda jednake veličine, u iQR kod stane više podataka. Dakle, za veliku količinu podataka je iQR kod bolja opcija nego standardni QR kod. Isto tako i za malu količinu podataka bi iQR kod bio optimalan, s obzirom na to da se minimalne dimenzije iQR koda sastoje od manje modula, nego minimalne dimenzije QR koda. Dakle, QR kod bi sadržavao više neiskorištenog prostora nego iQR kod.

Nadalje, iQR kod ima mogućnost biti u obliku pravokutnika što daje širi opseg primjene, s obzirom na to da se može nalaziti i na cilindričnim tijelima bez da deformacija cilindričnog tijela ometa čitanje.

Što se tiče ispravljanja pogrešaka, iQR kodovi imaju više razina ispravljanja pogrešaka što omogućava čitanje koda i prilikom većeg oštećenja. Ovo svojstvo je veoma korisno ako se iQR kodovi nalaze na mjestima gdje može doći do fizičkog oštećenja koda. Međutim, za svakodnevnu upotrebu nije potrebna razina koja omogućava ispraviti 50\% oštećenja. Najčešće korištena razina ispravljanja je M, dakle ispravljanje 15\% koda. 

Upotreba QR kodova je daleko rasprostranjenija nego upotreba iQR kodova. To bi se moglo objasniti time da za svakodnevni život nije potreban kapacitet iQR koda, niti mogućnost ispravljanja grešaka do 50\%. QR kodovi imaju primjenu u većini aspekata svakodnevnog života.

Zahvaljujući ovom načinu pohrane podataka, u mogućnosti smo pojednostaviti svakodnevni život. Brojne statistike pokazuju da se broj korisnika QR kodova povećava i da će vrlo vjerojatno tako i ostati, s obzirom na jednostavnost korištenja i raznoliku primjenu u gospodarstvu i svakodnevnom životu. 