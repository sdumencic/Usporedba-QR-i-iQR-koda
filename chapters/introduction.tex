\chapter*{Uvod} 
\addcontentsline{toc}{chapter}{Uvod}
QR kod označava matricu namijenjenu za pohranu podataka. Kratica QR dolazi od engleskih riječi "Quick response" (brzi odgovor) što dolazi od svojstva QR koda da je uređaj koji dekodira kod, u stanju to učiniti vrlo brzo. Matrica QR koda se sastoji od bijele pozadine te mreže kvadratastog oblika u kojoj se nalaze crni moduli - kvadratići. Svaki QR kod koji sadrži različite informacije ima različit poredak crnih kvadratića te čitač prema njima raspoznaje o kojim podacima se radi. 

Razvoj QR koda zahvaljujemo japanskoj kompaniji za proizvodnju motornih vozila (Denso Wave\footnote{Također, može se naći podatak da je tvrtka Toyota osmislila QR kod jer su Toyota i Denso Wave tada bile spojene kompanije}), koja je 1994. godine osmislila ovaj oblik dvodimenzionalnog barkoda kako bi pratili proizvodnju svojih motornih vozila. Zahvaljujući brzini skeniranja QR koda te količini pohranjenih informacija, ova vrsta zapisivanja podataka postajala je sve raširenija te se počela koristiti i u komercijalne svrhe. Neki od primjera pohranjivanja informacija podataka u QR kodu su: pohranjivanje URL adrese, teksta, predefinirane email ili SMS poruke, pohranjivanje lokacija na Google Maps-u, spajanje na wi-fi mrežu itd. Uporabom pametnih telefona se uporaba QR kodova znatno proširila te postala lako dostupna svima.

Sistem korištenja QR koda sastoji se od QR kodera i dekodera. Uloga kodera je da generira QR kod te da u njega pohrani informacije. Dok je uloga dekodera da informacije koje su pohranjene u QR kodu dekodira i prikaže.

Iz standardnog QR koda razvilo se je nekoliko varijanti. Primjeri varijanti su Micro QR kod, iQR kod, SQRC kod, FrameQR kod te HCC2D kod.

% \includegraphics[width=3cm]{images/standard_code.png}
% \includegraphics[width=3cm]{images/microqr.png}
% \includegraphics[width=3cm]{images/iQrCodeImage.png}
% \includegraphics[width=3cm]{images/sqrccode.jpg}
% \includegraphics[width=3cm]{images/frameqrcode.png}
% \includegraphics[width=3cm]{images/hcc2dcode.png}

\begin{figure}[!htpb]
	  \begin{center}
	   \subfloat[standardni QR kod]{\label{fig:a} \includegraphics[width=5cm,keepaspectratio=true]{images/standard_code.png}}
	   \subfloat[micro QR kod]{\label{fig:b}  %\\ % ukoliko se hoce iducu sliku u novi red
      \includegraphics[width=5.51cm,keepaspectratio=true]{images/microqr.png}}\\
        \subfloat[iQR kod]{\label{fig:c}  %\\ % ukoliko se hoce iducu sliku u novi red
      \includegraphics[width=5cm,keepaspectratio=true]{images/iqrcode.png}}
        \\
        \subfloat[frameQR kod]{\label{fig:d}  %\\ % ukoliko se hoce iducu sliku u novi red
      \includegraphics[width=5cm,keepaspectratio=true]{images/frameqrcode.png}}
        \subfloat[HCC2D kod]{\label{fig:e}  %\\ % ukoliko se hoce iducu sliku u novi red
      \includegraphics[width=5cm,keepaspectratio=true]{images/hcc2dcode.png}}
\caption{primjeri različitih vrsti QR kodova}
\label{fig:ID_slike}
	  \end{center}
\end{figure}
 


Na slikama vidimo razlike između navedenih varijanti te će razlike između standardnog i iQR koda biti detaljnije objašnjene.


